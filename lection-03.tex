\documentclass[aspectratio=169]{beamer}
\usepackage[utf8]{inputenc}
\usepackage[english,russian]{babel}
\usepackage{amssymb}
\usepackage{stmaryrd}
\usepackage{cmll}
\usepackage{xcolor}
\usepackage{proof}
\setbeamertemplate{navigation symbols}{}
\usepackage{tikz}
\usetikzlibrary{hobby,fit,backgrounds,calc,shapes.geometric,patterns}
\begin{document}

\newtheorem{axiom}{Аксиома}
\newtheorem{exmprus}{Пример}
\newtheorem{defrus}{Определение}
\newtheorem{lemmarus}{Лемма}
\newtheorem{thmrus}{Теорема}

\begin{frame}{}
\begin{center}\Large Немного об общей топологии.\end{center}
\end{frame}

\begin{frame}{Топологическое пространство}

\begin{defrus}Топологическим пространством называется упорядоченная пара $\langle X, \Omega \rangle$,
где $X$ --- некоторое множество, а $\Omega \subseteq \mathcal{P}(X)$, причём:
\begin{enumerate}
\item $\varnothing, X \in \Omega$
\item если $A_1, \dots, A_n \in \Omega$, то $A_1 \cap A_2 \cap \dots \cap A_n \in \Omega$;
\item если $\{A_\alpha\}$ --- семейство множеств из $\Omega$, то и $\bigcup_\alpha A_\alpha \in \Omega$.
\end{enumerate}

Множество $\Omega$ называется \emph{топологией}.
Элементы $\Omega$ называются открытыми множествами.
\end{defrus}

%\begin{defrus}
%Если $X \setminus A \in \Omega$, то $A$ --- замкнутое.
%\end{defrus}

\begin{defrus}$\mathcal{B}$ --- \emph{база} топологического пространства $\langle X, \Omega\rangle$ ($\mathcal{B} \subseteq \Omega$), 
если всевозможные объединения множеств из $\mathcal{B}$ дают $\Omega$.
\end{defrus}
\end{frame}

\begin{frame}{Примеры топологических пространств}
\begin{defrus}Эвклидово пространство (эквклидова топология) на $\mathbb{R}$: база топологии $\{(x,y)\ |\ x,y \in \mathbb{R}\}$.\end{defrus}
\begin{defrus}Дискретная топология: $\langle X, \mathcal{P}(X) \rangle$ --- все множества открыты.\end{defrus}
\begin{defrus}Топология стрелки: $\langle \mathbb{R}, \{(x,+\infty)\ |\ x\in\mathbb{R}\}\cup\{\varnothing,\mathbb{R}\}\rangle$ --- открыты все положительные лучи.\end{defrus}
\end{frame}

\begin{frame}{Подпространства и связные множества}
\begin{defrus}Пространство $\langle X_1, \Omega_1\rangle$ --- подпространство пространства $\langle X, \Omega\rangle$,
если $X_1 \subseteq X$ и $\Omega_1 = \{ A\cap X_1 | A \in \Omega\}$.
\end{defrus}
\begin{exmprus} $[0,1]$ с эвклидовой топологией на отрезке --- подпространство $\mathbb{R}$.
В нём множество $[0,0.5)$ открыто, так как $[0,0.5) = (-0.5,0.5) \cap [0,1]$.\end{exmprus}

\begin{defrus}Пространство $\langle X, \Omega\rangle$ связно, если нет $A,B \in \Omega$, что $A\cup B = X$,
$A \cap B = \varnothing$ и $A,B \ne \varnothing$.\end{defrus}

\begin{exmprus}Пространство $(0,1]\cup[2,3)$ в $\mathbb{R}$ несвязно: возьмём $A=(0,1]$ и $B = [2,3)$.

Дискретное топологическое пространство $\langle X, \mathcal{P}(X)\rangle$ несвязно при $|X| > 1$: пусть $a \in X$, 
тогда $A = \{a\}$ и $B = X \setminus A$.
\end{exmprus}
\end{frame}

\begin{frame}{Топология на деревьях}
\begin{defrus}Пусть некоторый лес задан конечным множеством вершин $V$ и
отношением $(\preceq)$ <<быть потомком>>. Тогда подмножество его вершин $X\subseteq V$ назовём открытым, 
если из $a \in X$ и $a \preceq b$ следует, что $b \in X$.\end{defrus}

\begin{exmprus}
\begin{center}\tikz[every fit/.style={trapezium,draw,inner sep=-2pt}]{
\node at (0,2.5)  (Title) {Открыты};
\node at (1,2)   (A) {$W_1$};
\node at (0,0.5) (B) {$W_2$};
\node at (1,0.5) (C) {$W_3$};
\node at (2,0.5) (D) {$W_4$};
\node at (0,-1) (E) {$W_5$};
\node at (2,-1) (F) {$W_6$};
\draw[->] (A) to (B); \draw[->] (B) to (E);
\draw[->] (A) to (C);
\draw[->] (A) to (D); \draw[->] (D) to (F);
  \begin{pgfonlayer}{background}
  \draw[red,fill=red,opacity=0.2](B.north west) 
     to[closed,curve through={
       (B.south west) ..
       (E.south west) .. (E.south east) .. ($(E.north east)!0.5!(C.south west)$) .. (C.south east)
     }] (C.north east);
  \draw[blue,fill=blue,opacity=0.2](D.north west) to 
       [closed, curve through={ (D.south west) .. (F.south west) .. (F.south east) .. (D.south east) }
       ] (D.north east);
  \end{pgfonlayer}
}
\tikz[every fit/.style={trapezium,draw,inner sep=-2pt}]{
\node at (0,2.8)  (Title) {Не открыты};
\node at (1,2)   (A) {$W_1$};
\node at (0,0.5) (B) {$W_2$};
\node at (1,0.5) (C) {$W_3$};
\node at (2,0.5) (D) {$W_4$};
\node at (0,-1) (E) {$W_5$};
\node at (2,-1) (F) {$W_6$};
\draw[->] (A) to (B); \draw[->] (B) to (E);
\draw[->] (A) to (C);
\draw[->] (A) to (D); \draw[->] (D) to (F);
  \begin{pgfonlayer}{background}
  \draw[brown,pattern color=brown!50,pattern=north east lines](A.north west) to 
       [closed, curve through={ (A.south west) .. (C.north west) .. (C.south west) .. (D.south east) }
       ] (A.north east);
  \draw[brown,pattern color=brown!50,pattern=north east lines](B.north west) to 
       [closed, curve through={ (B.south west) }
       ] (B.south east);
  \end{pgfonlayer}
}
\end{center}
\end{exmprus}
\end{frame}

\begin{frame}{Связность деревьев}

\begin{lemmarus}Лес связен (является одним деревом) тогда и только тогда, когда соответствующее ему 
топологическое пространство связно.\end{lemmarus}

\begin{proof}\begin{enumerate}\item Дерево связно: пусть не так и найдутся открытые непустые $A$,$B$, что 
$A \cup B = V$ и $A \cap B = \varnothing$. Пусть $v \in V$ --- корень
дерева и пусть $v \in A$ (для определённости). Тогда $A = \{ x \ |\ v \preceq x \}$ и $B = \varnothing$.
\item Пусть лес топологически связен, но есть несколько разных корней $v_1, v_2, \dots, v_k$. 
Возьмём $A_i = \{ x\ |\ v_i \preceq x \}$. Тогда все $A_i$ открыты, непусты, дизъюнктны и $V = \cup A_i$.
\end{enumerate}
\end{proof}
\end{frame}

\begin{frame}[fragile]{Пишем скобки или нет?}

Вы как пишете: $\sin x$ или $\sin (x)$? \pause

\begin{verbatim}
int main () {
    return sizeof 0;
}
\end{verbatim}
\pause

Соглашение о записи:
$$\text{sizeof}\ \varnothing = \text{sizeof}(\varnothing) = 0$$

но:
   $$\text{sizeof}\{\varnothing\} = \text{sizeof}(\{\varnothing\}) = 1$$

\end{frame}

\begin{frame}{Минимальные и максимальные элементы}

\begin{defrus}
Множество нижних граней $X\subseteq\mathcal{U}$: $\mbox{\upshape lwb}_\mathcal{U} X = \{ y\in \mathcal{U}\ |\ y \preceq x\text{ при всех } x \in X\}$.
Множество верхних граней $X\subseteq\mathcal{U}$: $\mbox{\upshape upb}_\mathcal{U} X = \{ y\in\mathcal{U}\ |\ x \preceq y \text{ при всех } x \in X\}$.
\end{defrus}

\begin{defrus}
\begin{tabular}{ll}
минимальный ($m \in X$): нет меньшего & при всех $y \in X$, $y \preceq m$ влечёт $y = m$ \\
максимальный ($m \in X$): нет большего & при всех $y \in X$, $m \preceq y$ влечёт $y = m$ \\
наименьший ($m \in X$): меньше всех & при всех $y \in X$ выполнено $m \preceq y$\\
наибольший ($m \in X$): больше всех & при всех $y \in X$ выполнено $y \preceq m$\\
инфимум: наибольшая нижняя грань & $\inf_\mathcal{U} X = \mbox{\upshape наиб}(\mbox{\upshape lwb}_\mathcal{U} X)$\\
супремум: наименьшая верхняя грань & $\sup_\mathcal{U} X = \mbox{\upshape наим}(\mbox{\upshape upb}_\mathcal{U} X)$
\end{tabular}
\end{defrus}

\begin{exmprus}\vspace{-0.5cm}
\begin{center}\tikz{
  \draw[-stealth, line width=1, color=gray] (-1,0) to (9,0);
  \foreach \n [evaluate=\n as \rev using {int(ln(1/\n+1)*1000)}] in {1,...,100} { \draw[line width=0.7,color=black!\rev] ({2+3/\n},-0.1) to ({2+3/\n},0.1); }
  \node (hdr) at (3.5,0.5) { $\left\{ \frac{1}{n}\ |\ n \in \mathbb{N}\right\}$ };
  \draw[circle] (2,0);
  \node[text height=3ex] (lwb) at (0.5,-0.3) {$\text{\upshape lwb}_\mathbb{R}$};
  \node[text height=3ex] (lwb) at (7.5,-0.3) {$\text{\upshape upb}_\mathbb{R}$};
  \foreach \x in {2,5} {
    \node[draw,isosceles triangle,rotate=90,fill=black,scale=0.3] (s\x) at (\x,-0.4) {};
  }
  \node[text height=3ex] (bsup) at (5,-0.7) {наиб, max, sup};
  \node[text height=3ex] (binf) at (2,-0.7) {inf};
  \path[pattern=north east lines] (-0.5,-0.1) rectangle (2,0.1);
  \path[pattern=north west lines] (5,-0.1) rectangle (8.5,0.1);
}\end{center}
\end{exmprus}
\end{frame}

\begin{frame}{Пример: делимость}
На $\mathbb{N}$ положим $a \preceq b$, если $b\ \raisebox{-0.75pt}{\vdots}\ a$. 
\begin{exmprus}Множество $\{2,3,6\}$\vspace{0.2cm}

\begin{tabular}{lll}
Минимальные: & 2,3 & $2\ \raisebox{-0.75pt}{\vdots}\ x$ влечёт $x = 1$ или $x = 2$, то же про 3\\
Наименьший: & отсутствует & $2 \not\preceq 3$ и $3 \not\preceq 2$\\
Инфимум: & 1 & $1 \preceq x$ при всех $x \in \mathbb{N}$
\end{tabular}
\end{exmprus}

\tikz{
  \draw[-stealth, line width=1, color=gray] (0,-0.5) to (7,-0.5);
  \foreach \i in {1,2,3,6} { \draw[-] (\i,-0.6) to (\i,-0.4); }
  \node[circle] (1) at (1,0) {$1$};
  \node[circle] (2) at (2,0) {$2$};
  \node[circle] (3) at (3,0) {$3$};
  \node[circle] (6) at (6,0) {$6$};
  \draw [dashed,rounded corners=4,gray,line width=0.5] (1.7,-0.3) rectangle (6.3,0.3);
  \foreach \b/\e in {1/2, 1/3, 2/6, 3/6} {
      \draw[stealth-, line width=1, color=black!50!green] (\b,-0.55) to [curve through={({(\b+\e)/2},{-0.55-(\e-\b)/5})}] (\e,-0.55);
  }

  \node[circle,inner sep=0.3] (1) at (10,-1.3) {$1$};
  \node[circle,inner sep=0.3] (2) at (9.5,-0.5) {$2$};
  \node[circle,inner sep=0.3] (3) at (10.5,-0.5) {$3$};
  \node[circle,inner sep=0.3] (6) at (10,0.3) {$6$};
  \draw [dashed,rounded corners=4,gray,line width=0.5] (9,-0.8) rectangle (11,0.6);
  \foreach \b/\e in {1/2, 1/3, 2/6, 3/6} {
      \draw[stealth-, line width=1, color=black!50!green] (\b) to (\e);
  }
}\pause
\vspace{-0.3cm}
\begin{exmprus}Множество $\{6\}$ --- минимальный, наименьший: 6 (так как $6 \preceq 6$)\\
Инфимум отсутствует: $\text{\upshape lwb}\{6\} = \{1,2,3\}$, $\text{\upshape наиб}\{1,2,3\}$ не определён.
\end{exmprus}

\end{frame}

\begin{frame}{Пример: внутренность множества}
\begin{defrus}[внутренность множества] Рассмотрим $\langle X, \Omega\rangle$ и возьмём $(\subseteq)$ как отношение частичного порядка на $\mathcal{P}(X)$.
Тогда $A^\circ := \inf_\Omega (\{ A\})$. %То есть, $A^\circ = \text{\upshape наиб}_\Omega\{ Q \in \Omega\ |\ Q \subseteq A\}$.
\end{defrus}

\begin{thmrus}$A^\circ$ определена для любого $A$.\end{thmrus}
\begin{proof}

Пусть $V = \text{lwb}_\Omega\{ A \} = \{ Q \in \Omega\ |\ Q \subseteq A\}$. Тогда $\inf_\Omega \{A\} = \bigcup V$.

Напомним, $\inf_\mathcal{U} T = \text{наиб}(\text{lwb}_\mathcal{U} T)$.

\begin{enumerate}
\item Покажем принадлежность: $\bigcup V \subseteq A$ и $\bigcup V \in \Omega$ как объединение открытых.
\item Покажем, что все из $V$ меньше или равны: пусть $X \in V$ то есть $V = \{ X, \dots \}$, тогда $X \subseteq X \cup \dots$, тогда $X \subseteq \bigcup V$
\end{enumerate}
\end{proof}
\end{frame}

\begin{frame}{Решётка}
\begin{defrus}Решёткой называется упорядоченная пара: $\langle X, (\preceq)\rangle$, 
где $X$ --- некоторое множество, а $(\preceq)$ --- частичный порядок на $X$, такой, 
что для любых $a,b \in X$ определены $a + b = \sup\{a,b\}$ и $a \cdot b = \inf\{a,b\}$.
\end{defrus}

То есть, $a + b$ --- наименьший элемент $c$, что $a \preceq c$ и $b \preceq c$.

\begin{exmprus}
$\langle\Omega, (\subseteq)\rangle$ --- решётка.
$\langle\mathbb{N}\setminus\{1\}, (\raisebox{-0.75pt}{\vdots})\rangle$ --- не решётка.
\end{exmprus}
\end{frame}

\begin{frame}{Псевдодополнение}
Псевдодополнением $a \rightarrow b$ называется наибольший из $\{ x \ |\ a \cdot x \preceq b\}$.

\begin{exmprus}\vspace{-0.3cm}
\begin{center}\tikz{
  \node[circle,inner sep=0.3] (A) at (1,-1.3) {$a$};
  \node[fill=cyan!0.2,circle,inner sep=0.3] (B) at (0.5,-0.5) {$b$};
  \node[circle,inner sep=0.3] (C) at (1.5,-0.5) {$c$};
  \node[circle,inner sep=0.3] (D) at (1,0.3) {$d$};
  \foreach \b/\e in {A/B, A/C, B/D, C/D} {
      \draw[stealth-, line width=1, color=black!50!green] (\b) to (\e);
  }

  \node (txt) at (4,-0.5) { \begin{tabular}{l} $a \cdot b = a$ \\ $b \cdot b = b$ \\ $c \cdot b = a$ \\ $d \cdot b = b$ \end{tabular} };
  \draw[fill=cyan,fill=cyan,opacity=0.2](C.north west) 
     to[closed,curve through={
       (A.south west) .. (A.south east)
     }] (C.north east);
}\end{center}

Здесь $b \rightarrow c = \text{наиб}\{x\ |\ b \cdot x \preceq c\} = \text{наиб}\{ a, c \} = c$
\end{exmprus}

\begin{exmprus}[нет псевдодополнения: диамант и пентагон]\vspace{-0.3cm}
\begin{center}\tikz{
  \node[circle,inner sep=0.3] (A) at (1,-1.3) {$a$};
  \node[circle,inner sep=0.3] (B) at (0,-0.5) {$b$};
  \node[circle,inner sep=0.3] (C) at (1,-0.5) {$c$};
  \node[circle,inner sep=0.3] (D) at (2,-0.5) {$d$};
  \node[circle,inner sep=0.3] (E) at (1,0.3) {$e$};
  \foreach \b/\e in {A/B, A/C, A/D, B/E, C/E, D/E} {
      \draw[stealth-, line width=1, color=black!50!green] (\b) to (\e);
  }
  \draw[pattern=north west lines,opacity=0.5,pattern color=olive](C.north west) 
     to[closed,curve through={
       (A.south west) .. (A.south east) .. (D.south east)
     }] (D.north east);
  \node (BC) at (1,-1.8) {$b \rightarrow c = \text{наиб}\{ a, c, d \}$};

  \node[circle,inner sep=0.3] (A1) at (9,-1.3) {$a$};
  \node[circle,inner sep=0.3] (B1) at (8,-1) {$b$};
  \node[circle,inner sep=0.3] (C1) at (8,0) {$c$};
  \node[circle,inner sep=0.3] (D1) at (10,-0.5) {$d$};
  \node[circle,inner sep=0.3] (E1) at (9,0.3) {$e$};
  \foreach \b/\e in {A1/B1, B1/C1, C1/E1, A1/D1, D1/E1} {
      \draw[stealth-, line width=1, color=black!50!green] (\b) to (\e);
  }

  \node (B1C1) at (9,-1.8) {$c \rightarrow b = \text{наиб}\{ a, b, d \}$};
  \draw[pattern=north west lines,opacity=0.5,pattern color=olive](B1.north west) 
     to[closed,curve through={
       (A1.south east) .. (D1.south east) .. (D1.north east)
     }] (B1.north east);
}\end{center}
\end{exmprus}
\end{frame}

\begin{frame}{Особые решётки}
\begin{defrus}Дистрибутивной решёткой называется такая, что для любых $a,b,c$ выполнено
$a \cdot (b + c) = a \cdot b + a \cdot c$.
\end{defrus}

\begin{defrus}Импликативная решётка --- такая, в которой для любых элементов есть псевдодополнение.\end{defrus}

\begin{lemmarus}Любая импликативная решётка --- дистрибутивна.\end{lemmarus}
\end{frame}

\begin{frame}{Ноль и один}
\begin{defrus}0 --- наименьший элемент решётки, а 1 --- наибольший элемент решётки\end{defrus}
\begin{lemmarus}В любой импликативной решётке $\langle X, (\preceq)\rangle$ есть 1\end{lemmarus}
\begin{proof} Рассмотрим $a \rightarrow a$, тогда $a \rightarrow a = \text{наиб}\{ c \ |\ a \cdot c \preceq a\} = 
\text{наиб} X = 1$.
\end{proof}
\begin{defrus}Импликативная решётка с 0 --- псевдобулева алгебра (алгебра Гейтинга).
В такой решётке определено $\sim a := a \rightarrow 0$ \end{defrus}
\begin{defrus}Булева алгебра --- псевдобулева алгебра, в которой $a\ + \sim a = 1$ для всех $a$.\end{defrus}
\end{frame}

\begin{frame}{Булева алгебра является булевой алгеброй в смысле решёток}
\begin{proof}
Символы булевой алгебры: $(\with),(\vee),(\neg),\text{Л},\text{И}$.\\
Символы решёток: $(+),(\cdot),(\rightarrow),(\sim),0,1$\\
Упорядочивание: $\text{Л} \le \text{И}$.

\begin{enumerate}
\item $a \with b = \min(a,b)$, $a \vee b = \max(a,b)$ 
(анализ таблицы истинности), отсюда $a \cdot b = a \with b$ и $a + b = a \vee b$.

\item $a \rightarrow b = \neg a \vee b$, так как:
$$a \rightarrow b = \text{наиб}\{ c | c \with a \le b\} = \left\{\begin{array}{ll}\neg a,& b = \text{Л}\\
                                                 \text{И},& b = \text{И}\end{array}\right.$$

\item $0 = \min\{\text{И},\text{Л}\} = \text{Л}$, $1 = \max\{\text{И},\text{Л}\} = \text{И}$, $\sim a = a \rightarrow 0 = \neg a \vee \text{Л} = \neg a$.
Заметим, что $a\ + \sim a = a \vee \neg a = \text{И}$.
\end{enumerate}
Итого: булева алгебра --- импликативная решётка с 0 и с $a\ + \sim a = 1$.

\end{proof}
\end{frame}

\begin{frame}{Множества и топологии как решётки}
\begin{lemmarus}
$\langle \mathcal{P}(X), (\subseteq) \rangle$ --- булева алгебра.
\end{lemmarus}

\begin{proof}
$a \rightarrow b = \text{наиб}\{ c \subseteq X\ |\ a \cap c \subseteq b\}$. Т.е. наибольшее, не содержащее точек из $a \setminus b$.
Т.е. $X \setminus (a \setminus b)$. То есть $(X \setminus a) \cup b$.

$a\ +\sim a = a \cup (X \setminus a) \cup \varnothing = X$
\end{proof}

\begin{lemmarus}
$\langle \Omega, (\subseteq) \rangle$ --- псевдобулева алгебра.
\end{lemmarus}

\begin{proof}
$a \rightarrow b = \text{наиб}\{ c \in \Omega\ |\ a \cap c \subseteq b\}$. 
Т.е. наибольшее открытое, не содержащее точек из $a \setminus b$.
То есть, $(X \setminus (a \setminus b))^\circ$. То есть, $((X \setminus a) \cup b)^\circ$.
\end{proof}

\end{frame}

\begin{frame}{Решётки и исчисление высказываний}
\begin{defrus}Пусть некоторое исчисление высказываний оценивается значениями из некоторой решётки.
Назовём оценку согласованной с исчислением, если 
$\llbracket\alpha\with\beta\rrbracket = \llbracket\alpha\rrbracket\cdot\llbracket\beta\rrbracket$,
$\llbracket\alpha\vee\beta\rrbracket = \llbracket\alpha\rrbracket+\llbracket\beta\rrbracket$,
$\llbracket\alpha\rightarrow\beta\rrbracket = \llbracket\alpha\rrbracket\rightarrow\llbracket\beta\rrbracket$,
$\llbracket\neg\alpha\rrbracket =\ \sim\llbracket\alpha\rrbracket$,
$\llbracket A \with\neg A\rrbracket = 0$, $\llbracket A\rightarrow A \rrbracket = 1$.
\end{defrus}

\begin{thmrus}Любая псевдобулева алгебра, являющаяся согласованной оценкой интуиционистского исчисления высказываний,
является его корректной моделью: если $\vdash\alpha$, то $\llbracket\alpha\rrbracket = 1$.
\end{thmrus}

\begin{thmrus}Любая булева алгебра, являющаяся согласованной оценкой классического исчисления высказываний, 
является его корректной моделью: если $\vdash\alpha$, то $\llbracket\alpha\rrbracket = 1$
\end{thmrus}

\end{frame}

\begin{frame}{Алгебра Линденбаума}
\begin{defrus}Определим предпорядок на высказываниях: $\alpha \preceq \beta := \alpha \vdash \beta$ в интуиционистском исчислении высказываний.
Также $\alpha\approx\beta$, если $\alpha\preceq\beta$ и $\beta\preceq\alpha$.\end{defrus}
\begin{defrus}Пусть $L$ --- множество всех высказываний. Тогда алгебра Линденбаума $\mathcal{L} = L/_\approx$.\end{defrus}
\begin{thmrus}$\mathcal{L}$ --- псевдобулева алгебра.\end{thmrus}
\begin{proof}[Схема доказательства] Надо показать, что $(\preceq)$ есть отношение порядка на $\mathcal{L}$, что
$[\alpha\vee\beta]_\mathcal{L} = [\alpha]_\mathcal{L}+[\beta]_\mathcal{L}$, 
$[\alpha\with\beta]_\mathcal{L}[\alpha]_\mathcal{L} = \cdot[\beta]_\mathcal{L}$, импликация есть псевдодополнение,
$[A\with\neg A]_\mathcal{L} = 0$, $[\alpha]_\mathcal{L}\rightarrow 0 = [\neg\alpha]_\mathcal{L}$.\end{proof}
\end{frame}

\begin{frame}{Полнота псевдобулевых алгебр}
\begin{thmrus}Пусть $\llbracket\alpha\rrbracket = [\alpha]_\mathcal{L}$.
Такая оценка интуиционистского исчисления высказываний алгеброй Линденбаума является согласованной.
\end{thmrus}

\begin{thmrus}Интуиционистское исчисление высказываний полно в псевдобулевых алгебрах:
если $\models\alpha$ во всех псевдобулевых алгебрах, то $\vdash\alpha$. \end{thmrus}
\begin{proof}Возьмём в качестве модели исчисления алгебру Линденбаума: 
$\llbracket \alpha \rrbracket = [\alpha]_\mathcal{L}$. 

Пусть $\models\alpha$. Тогда $\llbracket\alpha\rrbracket = 1$ во всех псевдобулевых алгебрах, в том числе
и $\llbracket\alpha\rrbracket = 1_\mathcal{L}$. То есть $[\alpha]_\mathcal{L} = [A\rightarrow A]_\mathcal{L}$.
То есть $A \rightarrow A \approx \alpha$. Значит, в частности, $A \rightarrow A \vdash \alpha$. 
Значит, $\vdash\alpha$.\end{proof}
\end{frame}

\end{document}