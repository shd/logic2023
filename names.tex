\documentclass[11pt,a4paper,oneside]{scrartcl}
\usepackage[utf8]{inputenc}
\usepackage[english,russian]{babel}
\usepackage[top=1cm,bottom=1cm,left=1cm,right=1cm]{geometry}

\begin{document}
\pagestyle{empty}

\begin{center}
{\large\scshape\bfseries Курс <<Математическая логика>>}\\
{\large\scshape Список имён и терминов для диктанта.}\\
\itshape ИТМО, группы M3232--M3239, весна 2023 г.
\end{center}

%\vspace{0.3cm}

Вам будет предложено записать на слух некоторые термины из списка, приведённого ниже 
(конкретные пункты списка выбирает экзаменатор). При возможности следует записать полный вариант 
по краткому названию: например, <<интерпретация логических связок Брауэра-Гейтинга-Колмогорова>> 
по <<BHK-интерпретация>>.

Выполнение данного задания (все термины полностью записаны в правильной орфографии) 
оценивается в один балл.

\begin{enumerate}
\item Интуиционистская логика, сколемизация, предварённая нормальная форма.
\item Конъюнкция и дизъюнкция, ДНФ (дизъюнктивная нормальная форма).
\item BHK-интерпретация (интерпретация логических связок Брауэра-Гейтинга-Колмогорова).
\item Импликация, антецедент, сукцедент.
\item Условия доказуемости Гильберта-Бернайса-Лёба.
\item Алгебра Линденбаума, теорема Лёвенгейма-Сколема.
\item Теорема Чёрча-Россера, изоморфизм Карри-Ховарда.
\item Парадокс Бурали-Форте, аксиоматика ZF (Цермело-Френкеля).
\item Общезначимость, универсум Эрбрана, теорема Диаконеску.
\end{enumerate}
\end{document}
